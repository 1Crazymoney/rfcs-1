\documentclass[11pt,twoside,a4paper]{article}
\usepackage{tikz}

\usepackage{bera}% optional: just to have a nice mono-spaced font
\usepackage{listings}
\usepackage{xcolor}

\colorlet{punct}{red!60!black}
\definecolor{background}{HTML}{EEEEEE}
\definecolor{delim}{RGB}{20,105,176}
\colorlet{numb}{magenta!60!black}

\lstdefinelanguage{json}{
    basicstyle=\normalfont\ttfamily,
%    numbers=left,
%    numberstyle=\scriptsize,
%    stepnumber=1,
%    numbersep=8pt,
%    showstringspaces=false,
%    breaklines=true,
%    frame=lines,
%    backgroundcolor=\color{background},
%    literate=
%     *{0}{{{\color{numb}0}}}{1}
%      {1}{{{\color{numb}1}}}{1}
%      {2}{{{\color{numb}2}}}{1}
%      {3}{{{\color{numb}3}}}{1}
%      {4}{{{\color{numb}4}}}{1}
%      {5}{{{\color{numb}5}}}{1}
%      {6}{{{\color{numb}6}}}{1}
%      {7}{{{\color{numb}7}}}{1}
%      {8}{{{\color{numb}8}}}{1}
%      {9}{{{\color{numb}9}}}{1}
%      {:}{{{\color{punct}{:}}}}{1}
%      {,}{{{\color{punct}{,}}}}{1}
%      {\{}{{{\color{delim}{\{}}}}{1}
%      {\}}{{{\color{delim}{\}}}}}{1}
%      {[}{{{\color{delim}{[}}}}{1}
%      {]}{{{\color{delim}{]}}}}{1},
}
\lstset{language=json}

\begin{document}
\title{{\em ILP v4}: Version 4 of the Interledger Protocol.}
\author{Dennis Appelt, Adrian Hope-Bailie, Michiel de Jong, Evan Schwartz, Ben Sharafian, Stefan Thomas, Bob Way}
\date{April 2018}
\maketitle
\begin{abstract}
In 2015, Evan Schwartz and Stefan Thomas published version 1 of the Interledger protocol (ILP). After three years of research,
we are proud to present its evolution, version 4. ILPv4 uses a chain of Hash Time Lock Agreements (transfers) between network neighbors, to form
multi-hop Interledger payments that remotely interconnect non-neighbors.
\end{abstract}

\section{Conditional Transfers between Neighbors}
Consider a network of agents, where each agent has accounting relationships with at least one other agent in the network. An accounting
relationship between two agents is defined by a unit of value, and a log of Interledger transfers, which together define what one agent owes the other.

\subsection{Hash Time Lock Agreements}
Suppose Alice and Bob are two agents who are neighbors in an Interledger network. A Hash Time Lock Agreement (HTLA) between them is then
defined by a {\tt hash}, a {\tt time},  and an agreed {\tt amount}. If Alice sends such a HTLA to Bob, and Bob rejects the agreement, the state of their accounting
relationship does not change. However, if Bob presents the SHA256 preimage of the {\tt hash} before {\tt time}, then the HTLA says Alice owes {\tt amount} to Bob.

\subsection{Interledger Transfers}
An Interledger Transfer is a specific kind of Hash Time Lock Agreement, designed to form multi-hop chains of HTLAs in the network. To this end,
it specifies a {\tt destination}, which is an ILP address: a hierarchical identifier for the agent in the network who may be able to produce the SHA256 preimage of the
HTLA's hash. Apart from this destination field, Interledger transfers have {\tt data} fields, which should be transported end-to-end over a multi-hop chain (see below).

\section{Multi-hop Hashlocks}
Suppose Alice sends an Interledger Transfer to Bob, whose {\tt destination} points at one of Bob's other network neighbors, Charlie. Bob can then create a second
transfer, from him to Charlie, with the same {\tt hash}, and with a slightly earlier {\tt time} deadline. If Charlie fulfills this transfer from Bob, Bob will be able
to use the preimage presented by Charlie, to fulfill the incoming transfer from Alice. We call this a two-hop transaction, from Alice, via Bob, to Charlie.

\section{ILP Addresses}
As stated earlier, the {\tt destination} field in a Prepare packet, and the {\tt triggeredBy} field in a Reject packet, are ILP addresses, a hierarchical identifier for an agent in the network.
An ILP address consists of one or more segments, separated by a dot ({\tt .}). Each segment may contain alphanumeric characters (upper or lower case, case-sensitive),
underscores ({\tt _}),
tildes ({\tt ~|),
and hyphens ({\tt -}).

\subsection{Static Routing}
In Interledger networks where the set of agents participating in the network and the accounting relationships between them don't change,
an easy way to make sure that all agents in the network know how to route multi-hop payments towards their destination, would be to configure each agent with static information about the
network topology. 
In any case, the first two agents to form the network can choose their own identifiers freely, as long as they are different from each other. Then a new agent joins, one of two things can happen:
they may become a child of the agent through which they get joined to the network, or become a first-class citizen.

\subsection{Sub-Addresses}
If the new agent becomes a child, then it takes the address of the parent through which it joins the network, and add one or more segments at the end. By using its parent's address as a
prefix, other agents in the network will always be able to route payments to the new agent, provided they know how to route a payment to the new agent's parent.

\subsection{Route Broadcasts}
If the new agent wants to become a first-class citizen of the network, they can pick their ILP address at will, and send their neighbor a message to broadcast the address they picked.
If that address already exists in the network, the neighbor should respond with an error message. Otherwise, they can forward the broadcast to inform all existing agents of the agent
that newly joined the network.

\section{Interledger Packets}
An Interledger Packet is a standardized message, which two neighbors can use for communicating about Interledger transfers. There are three Interledger Packet
types, {\tt Prepare} (for proposing a transfer), {\tt Fulfill} (for accepting a transfer from a neighbor), and {\tt Reject} (for explicitly rejecting it before the deadline).

\subsection{End-to-end Data}
As multiple transfers are chained into an Interledger payment, the initiator of the first transfer in the chain (the 'sender') may want to send some data to the
beneficiary of the last transfer in the chain (the 'receiver'). For this, Prepare packets have a {\tt data} field. Each agent who participates in a chain of Interledger
transfers is expected to copy this data unchanged from their incoming transfer's Prepare packet, to their outgoing transfer's Prepare packet. Likewise,
just like one {\tt Prepare} packet triggers the next, and its data is carried over, each {\tt Fulfill} packet will generally trigger a {\tt Fulfill} packet, and each {\tt Reject}
packet will trigger another {\tt Reject} packet, on the return path. For these, the {\tt data} field should also be copied over, so that the receiver can relay data back
to the sender. This transportation of end-to-end data is not guaranteed by the chain of hashlock conditions, but can still be useful if sender and receiver sign
the data payload cryptographically.

\subsection{Prepare}
A Prepare packet contains five fields: {\tt amount} (from the HTLA), {\tt expiresAt} (the HTLA deadline), {\tt executionCondition} (the HTLA hash), {\tt destination} (the ILP address
of the payment's receiver, who may be able to present the SHA256 preimage of the {\tt executionCondition}), and {\tt data} (to be copied to the next transfer in the payment
chain, to enable end-to-end communication from sender to receiver). These five fields are encoded in an OER sequence as a {\tt UInt64}, a {\tt PrintableString (size 17)}, a {\tt UInt256},
an {\tt IA5String (size 1..1023)}, and an {\tt OCTET STRING (size 0..32767)}, in that order (see https://github.com/interledger/rfcs/blob/master/asn1/InterledgerProtocol.asn#L16-L31).

\subsection{Fulfill}
A Fulfill packet is an OER sequence of two fields: {\tt fulfillment}, a {\tt UInt256} for the SHA256 preimage of the {\tt executionCondition} from the transfer's Prepare packet,
and {\tt data} (to be passed back unaltered from receiver to sender), an {\tt OCTET STRING (size 0..32767)}.

\subsection{Reject}
A Reject packet is an OER sequence of four fields:
}code} (an {\tt IA5String (size 3)} from the list of Interledger Error Codes, see https://interledger.org/rfcs/0027-interledger-protocol-4/#error-codes),
}triggeredBy} (}IA5String (size 1..1023)}, the ILP address of the agent who triggered this rejection),
}message} (}Ia5String (size 0..8191)}, a human-readable error message),
and {\tt data} (to be passed back unaltered from receiver to sender, an {\tt OCTET STRING (size 0..32767)}.

\subsection{Timing Disputes}
If Bob sends the Fulfill message before the {\tt expiresAt} timestamp from the corresponding Prepare message, but Alice only receives that message after the
timestamp, they would likely disagree on whether that message was sent on time. In case of such timing disputes, the party sending the Fulfill
packet is always given the benefit of the doubt. Note that this allows them to be owed the amount of the transfer, but they still rely on the other party's
cooperation to actually net this debt against a transfer in the opposite direction, so it's probably not a lucrative way for one agent to steal from the other.

\subsection{Idempotency}
In the communication channel between two neighbors,
a unique identifier should be assigned to each transfer. Each transfer starts with one participant (the Preparer) sending a Prepare packet.

\subsubsection{Using Message Acknowledgement}
If the communication channel supports delivery acknowledgement for messages, then the Preparer will keep repeating the Prepare packet, with its unique transfer identifier,
until an acknowledgement is received.
If the other participant can present the fulfillment on time, they will send a Fulfill packet, and keep repeating it with that same transfer identifier, until an acknowledgement is received.
If the other participant cannot present the fulfillment on time, they will send a Reject packet as soon as possible, and keep repeating it with the transfer identifier,
until an acknowledgement is received.

\subsubsection{Without Message Acknowledgment}
If the communication channel between two neighbors does not support delivery acknowledgment, then the Preparer should repeat the Prepare packet until either a Fulfill or Reject
packet for the same unique transfer identifier is received. When the Preparer stops repeating the Prepare packet, the Fulfiller or Rejecter can conclude that their response packet
was delivered successfully.

\bibliography{whitepaper}
\bibliographystyle{plain}
\end{document}
